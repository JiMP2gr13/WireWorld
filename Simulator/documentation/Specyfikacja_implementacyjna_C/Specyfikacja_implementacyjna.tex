\documentclass[a4paper,11pt, notitlepage ]{article}
\usepackage[T1]{fontenc}
\usepackage[polish]{babel}
\usepackage[utf8]{inputenc}
\usepackage{lmodern}
\usepackage{enumitem}
\usepackage{indentfirst}
\usepackage{graphicx}
\usepackage{wrapfig}
\usepackage{fancyhdr}
\usepackage{lastpage}
\pagestyle{fancy}
\fancyhf{}
\setcounter{page}{1}
\rfoot{Strona \thepage \hspace{1pt} z \pageref{LastPage}}
\selectlanguage{polish}
\makeatletter
\newcommand{\linia}{\rule{\linewidth}{0.4mm}}
\renewcommand{\maketitle}{\begin{titlepage}
    \vspace*{1cm}
    \begin{center}\small
    Politechnika Warszawska\\
    Wydział Elektryczny
    \end{center}
    \vspace{3cm}
    \noindent\linia
    \begin{center}
      \LARGE \textsc{\@title}
         \end{center}
     \linia
    \vspace{0.5cm}
    \begin{flushright}
    \begin{minipage}{8cm}
    \textit{\small Autorzy:}\\
    \normalsize \textsc{\@author} \par
    \end{minipage}
    \end{flushright}
    \vspace*{\stretch{6}}
    \begin{center}
    \@date
    \end{center}
  \end{titlepage}%
}
\makeatother
\author{J.~Korczakowski, nr albumu 291079\\ B.~Suchocki, nr albumu 291111\\ Grupa projektowa nr 11}
\title{Specyfikacja implementacyjna symulatora automatu komórkowego \textsl{Life}}
\frenchspacing
\begin{document}
\maketitle
\setcounter{page}{2}
\tableofcontents
\newpage
\section{Diagram programu}
\begin{figure}[h]
\centering
\includegraphics[width=13cm]{Modules_diagram}
\caption{Diagram}
\end{figure}

Na diagramie naszego projektu przedstawiliśmy wszystkie moduły, które wchodzą w jego skład. Moduły korzystają z funkcji tych modułów, które mają strzałki skierowane w ich kierunku.
\section{Opis modułów i ich funkcji}\label{Opis}
\subsection{Board}
Zawiera strukturę przechowywującą planszę i rozłożone na niej komórki oraz następujące funkcje:
\begin{enumerate}

\item \begin{verbatim} board_t make_board(int height, int width). \end{verbatim}
\begin{itemize}
\item Argumenty:
\begin{itemize}
\item \verb+height+, \verb+width+ - wysokość i szerokość planszy.
\end{itemize}
\item Wartość zwracana:
\begin{itemize}
\item Nowa plansza (obiekt struktury \verb+board_t+)
\end{itemize}
\item Działanie:
\begin{itemize}
\item Funkcja alokuje pamięć na strukturę \verb+board_t+ i zwraca nowo utworzony obiekt.
\end{itemize}
\end{itemize}


\item \begin{verbatim} board_t read_board_file(char* file_name). \end{verbatim}
\begin{itemize}
\item Argumenty:
\begin{itemize}
\item \verb+filename+ - nazwa pliku konfiguracyjnego wejciowego.
\end{itemize}
\item Wartość zwracana:
\begin{itemize}
\item Plansza reprezentująca stan generacji opisanej w pliku konfiguracyjnym wejściowym(obiekt struktury \verb+board_t+)
\end{itemize}
\item Działanie:
\begin{itemize}
\item Funkcja wywołuje \verb+make_board+ i do nowo stworzonej struktury przepisuje stany kolejnych komórek ('0' w pliku konfiguracyjnym oznacza komórkę martwą, a '1' żywą).
\end{itemize}
\end{itemize}




\item \begin{verbatim} board_t free_board(board_t b). \end{verbatim}
\begin{itemize}
\item Argumenty:
\begin{itemize}
\item \verb+b+ - plansza wskazująca na pamięć, którą program zwolni.
\end{itemize}
\item Wartość zwracana:
\begin{itemize}
\item Null
\end{itemize}
\item Działanie:
\begin{itemize}
\item Funkcja zwalnia pamięć wskazywaną przez argument wywołania.
\end{itemize}
\end{itemize}


\item \begin{verbatim} void print_board(FILE* out, board_t b). \end{verbatim}
\begin{itemize}
\item Argumenty:
\begin{itemize}
\item \verb+out+ - strumień wyjściowy, na który program wypisze reprezentację planszy.
\item \verb+b+ - plansza, której reprezentację chcemy wypisać na podany strumień wyjściowy.
\end{itemize}
\item Wartość zwracana:
\begin{itemize}
\item Void
\end{itemize}
\item Działanie:
\begin{itemize}
\item Funkcja wypisuje reprezentację planszy na zadany strumień wyjściowy.
\end{itemize}
\end{itemize}
\end{enumerate}


\subsection{Simulator}
Odpowiada za symulację kolejnych generacji komórek. Zawiera następujące funkcje:
\begin{enumerate}

\item \verb+board_t simulate_generation(board_t previous_generation,+ \\\verb+int rules[],int neighbourhood).+
\begin{itemize}
\item Argumenty:
\begin{itemize}
\item \verb+previous_generation+ - aktualny stan planszy,
\item \verb+rules+ - zasady życia komórek. Każda pozycja tablicy rules odpowiada ilości żywych sąsiadów, dla której komórka przechodzi do odpowiedniego stanu. Każda wartość na danej pozycji informuje do jakiego stanu przejdzie komórka z daną ilością żywych sąsiadów (liczbie żywych sąsiadów odpowiada pozycja w tablicy): 
\begin{itemize}
	\item przeżywa lub rodzi się w zależności od bieżącego stanu (3),
	\item rodzi się (2), 
	\item przeżywa (1), 
	\item na pewno umiera (0).
\end{itemize}
Jeśli komórka żyje i ma liczbę żywych sąsiadów odpowiadającą tylko rodzeniu się, umiera z zatłoczenia (stąd dodatkowa wartość: "przeżywa lub rodzi się").
\item \verb+neighbourhood+ - przyjęta zasada sąsiedztwa, czyli ilość rozpatrywanych sąsiadów (4 - sąsiedztwo von Neumanna, 8 - sąsiedztwo Moore'a).
\end{itemize}
\item Wartość zwracana:
\begin{itemize}
\item Plansza reprezentująca stan nowej generacji po przeprowadzeniu symulacji (obiekt struktury \verb+board_t+)
\end{itemize}
\item Działanie:
\begin{itemize}
\item Funkcja na podstawie podanych argumentów, symuluje przejcie komórek do następnej generacji. Symulacja wykonywana jest poprzez iteracyjne sprawdzanie ilości żywych sąsiadów 
(w podanej jako argument planszy reprezentującej stan generacji przed symulacją) i ustalanie (na podstawie danych zawartych w zasadach podanych jako argument wywołania funkcji) czy komórka przeżywa, umiera lub się ożywa.
Nowy stan każdej komórki zapisywany jest do nowej planszy.
\end{itemize}
\end{itemize}


\item \begin{verbatim} int count_neighbour_alive(int x,int y, int neighbourhood). \end{verbatim}
\begin{itemize}
\item Argumenty:
\begin{itemize}
\item \verb+x,y+ - współrzędne komórki, dla której zliczana jest iloć żywych sąsiadów,
\item neighbourhood - przyjęta zasada sąsiedztwa.
\end{itemize}
\item Wartość zwracana:
\begin{itemize}
\item Ilość żywych sąsiadów komórki o współrzędnych podanych jako argumenty wywołania funkcji.
\end{itemize}
\item Działanie:
\begin{itemize}
\item Funkcja zlicza ilość żywych sąsiadów komówki o podanym położeniu według zadanych zasad sąsiedztwa (Moore'a lub von Neumanna).
\end{itemize}
\end{itemize}

\item \begin{verbatim} int what_happens_with_cell(int neighbours_alive, int rules[]). \end{verbatim}
\begin{itemize}
\item Argumenty:
\begin{itemize}
\item \verb+neighbours_alive+ - ilość żywych sąsiadów komórki,
\item \verb+rules+ - przyjęte zasady przżywania komórek.
\end{itemize}
\item Wartość zwracana:
\begin{itemize}
\item Liczba określająca czy komórka przeżywa (2), umiera (0) czy ożywa (1).
\end{itemize}
\item Działanie:
\begin{itemize}
\item Funkcja na podstawie podanej ilości żywych sąsiadów ustala (korzystając z przyjętych zasad przeżywania komórek) czy komórka przeżyje, umrze lub ożyje.
\end{itemize}
\end{itemize}

\end{enumerate}

\subsection{Png generator}
Odpowiada za generowanie obrazów \verb+.png+ przedstawiających stan kolejnych generacji. Zawiera następujące funkcje:
\begin{enumerate}
\item \begin{verbatim} void process_file(board_t b). \end{verbatim}
\begin{itemize}
\item Argumenty:
\begin{itemize}
\item \verb+b+ - plansza reprezentująca stan generacji, dla której wygenerowany zostanie obraz \verb+.png+.
\end{itemize}
\item Wartość zwracana:
\begin{itemize}
\item Void
\end{itemize}
\item Działanie:
\begin{itemize}
\item Funkcja ustawia parametry obrazu wynikowego na podstawie podanej jako argument planszy (stanu generacji).
\end{itemize}
\end{itemize}


\item \begin{verbatim} void genarate(char* file_name). \end{verbatim}
\begin{itemize}
\item Argumenty:
\begin{itemize}
\item \verb+file_name+ - nazwa pliku obrazu wyjściowego \verb+.png+. 
\end{itemize}
\item Wartość zwracana:
\begin{itemize}
\item Void
\end{itemize}
\item Działanie:
\begin{itemize}
\item Funkcja generuje obraz \verb+.png+ o podanej jako argument wywołania nazwie.
\end{itemize}
\end{itemize}

\end{enumerate}

\subsection{Main}
Jest modułem sterującym programu. Zawiera następujące funkcje:
\begin{enumerate}

\item \verb+ void read_rules(FILE* settings_file, int rules[],+ \\\verb+ int neighbourhood)+
\begin{itemize}
\item Argumenty:
\begin{itemize}
\item \verb+settings_file+ - plik ustawień opisujący ustawienia gry w~życie,
\item \verb+rules+ - tablica, w której przechowywane będą zasady przeżywania komórek,
\item \verb+neighbourhood+ - zmienna, w której przechowywana będzie przyjęta zasada sąsiedztwa.
\end{itemize}
\item Wartość zwracana:
\begin{itemize}
\item Void
\end{itemize}
\item Działanie:
\begin{itemize}
\item Funkcja czyta z pliku zasady sąsiedztwa i umieszcza je w podanej jako argument zmiennej \verb+neighbourhood+.
Oprócz tego czyta zasady przeżywania komórek i zapisuje je do podanej jako argument tablicy \verb+rules+. 
\end{itemize}
\end{itemize}


\item \begin{verbatim} int main(int argc, char** argv) \end{verbatim}
\begin{itemize}
\item Argumenty:
\begin{itemize}
\item \verb+argc+ - ilość podanych argumentów w wierszu poleceń ,
\item \verb+argv+ - argumenty podane przez użytkownika w wierszu poleceń.
\end{itemize}
\item Wartość zwracana:
\begin{itemize}
\item Liczba całkowita. W przypadku błędu: 1, przy poprawnym wywołaniu: 0.
\end{itemize}
\item Działanie:
\begin{itemize}
\item Funkcja steruje wykonywaniem całego programu. Wywołuje funkcje z innych modułów, które razem tworzą spójne działanie. 
\end{itemize}
\end{itemize}

\end{enumerate}

\subsection{Board test}
Jest modułem testującym działanie modułu \verb+Board+. Zawiera następujące funkcje:
\begin{enumerate}

\item \begin{verbatim} void should_print_board(FILE* out, board_t b) \end{verbatim}
\begin{itemize}
\item Argumenty:
\begin{itemize}
\item \verb+out+ - strumień wyjściowy, na który powinna zostać wypisana plansza,
\item \verb+b+ - plansza, która powinna zostać wypisana.
\end{itemize}
\item Wartość zwracana:
\begin{itemize}
\item Void.
\end{itemize}
\item Działanie:
\begin{itemize}
\item Funkcja powinna poprawnie wypisać podaną jako argument planszę na dany strumień wyjściowy za pomocą funkcji \\\verb+print_board+ z modułu \verb+Board+. 
\end{itemize}
\end{itemize}


\item \begin{verbatim} void should_show_board_for_correct_board_file(char *file_name) \end{verbatim}
\begin{itemize}
\item Argumenty:
\begin{itemize}
\item \verb+file_name+ - nazwa pliku, z którego powinna zostać wczytana plansza. 
\end{itemize}
\item Wartość zwracana:
\begin{itemize}
\item Void.
\end{itemize}
\item Działanie:
\begin{itemize}
\item Funkcja powinna poprawnie przeczytać planszę z podanego (poprawnego) pliku wejściowego, korzystając z funkcji \\\verb+read_from_board_file+ z modułu \verb+Board+ oraz wypisać ją na standardowe wyjście.
\end{itemize}
\end{itemize}


\item \verb+ void should_show_error_message_for_incorrect_board_file+ \\\verb+(char *file_name)+
\begin{itemize}
\item Argumenty:
\begin{itemize}
\item \verb+file_name+ - nazwa błędnego pliku konfiguracyjnego wejściowego.
\end{itemize}
\item Wartość zwracana:
\begin{itemize}
\item Void.
\end{itemize}
\item Działanie:
\begin{itemize}
\item Funkcja próbuje przeczytać planszę z niepoprawnego pliku wejściowego, używając funkcji \verb+read_from_board_file+ z modułu \verb+Board+. Próba ta powinna zakończyć się niepowodzeniem. W takim przypadku zostanie wypisany komunikat o~błędzie.
\end{itemize}
\end{itemize}


\item \verb+ void should_show_confirmation_for_succesful_memory_+\\\verb+allocation(int height, int width)+
\begin{itemize}
\item Argumenty:
\begin{itemize}
\item \verb+height, width+ - wysokość i szerokość planszy, dla której alokowana będzie pamięć.
\end{itemize}
\item Wartość zwracana:
\begin{itemize}
\item Void.
\end{itemize}
\item Działanie:
\begin{itemize}
\item Funkcja próbuje zaalokować pamięć na planszę korzystając z funkcji \verb+make_matrix+ z modułu \verb+Board+. Jeżeli ta próba zakończy się powodzeniem, funkcja wypisze komunikat: "Udalo sie".
\end{itemize}
\end{itemize}

\item \verb+void should_show_error_message_for_unsuccessful_memory_+ \\\verb+allocation(int height, int width)+
\begin{itemize}
\item Argumenty:
\begin{itemize}
\item \verb+height, width+ - wysokość i szerokość planszy, dla której alokowana będzie pamięć. 
\end{itemize}
\item Wartość zwracana:
\begin{itemize}
\item Void.
\end{itemize}
\item Działanie:
\begin{itemize}
\item Funkcja próbuje zaalokować pamięć na planszę korzystając z~funkcji \verb+make_matrix+ z modułu \verb+Board+. Jeżeli ta próba zakończy się niepowodzeniem, funkcja wypisze komunikat o~błędzie. Przy wywoływaniu tej funkcji, jako 
argumenty podawane będą bardzo duże liczby w celu przetestowania zachowania funkcji przy braku potrzebnej pamięci. 
\end{itemize}
\end{itemize}


\item \begin{verbatim} int main(int argc, char** argv) \end{verbatim}
\begin{itemize}
\item Argumenty:
\begin{itemize}
\item \verb+argc+ - ilość podanych argumentów w wierszu poleceń,
\item \verb+argv+ - argumenty podane przez użytkownika w wierszu poleceń.
\end{itemize}
\item Wartość zwracana:
\begin{itemize}
\item Liczba '0'.
\end{itemize}
\item Działanie:
\begin{itemize}
\item Funkcja steruje działaniem modułu \verb+Board_test+. Wywołuje funkcje testowe.
\end{itemize}
\end{itemize}


\end{enumerate}




\subsection{Simulation test}
Jest modułem testującym działanie modułu \verb+Simulator+. Zawiera następujące funkcje:
\begin{enumerate}

\item \verb+ void should_show_generation_after_simulation(board_t+ \\\verb+ prevoius_generation, int rules[], int neighbourhood) +
\begin{itemize}
\item Argumenty:
\begin{itemize}
\item \verb+previous_generation+ - aktualny stan planszy,
\item \verb+rules+ - zasady życia komórek,
\item \verb+neighbourhood+ - przyjęta zasada sąsiedztwa.
\end{itemize}
\item Wartość zwracana:
\begin{itemize}
\item Void.
\end{itemize}
\item Działanie:
\begin{itemize}
\item Funkcja przeprowadza symulację jednej generacji komórek korzystając z funkcji \verb+simulate_generation+ z modułu \verb+Simulator+. Wynikową generację komórek wypisuje na standardowe wyjście.
\end{itemize}
\end{itemize}



\item \begin{verbatim} int main(int argc, char** argv) \end{verbatim}
\begin{itemize}
\item Argumenty:
\begin{itemize}
\item \verb+argc+ - ilość podanych argumentów w wierszu poleceń ,
\item \verb+argv+ - argumenty podane przez użytkownika w wierszu poleceń.
\end{itemize}
\item Wartość zwracana:
\begin{itemize}
\item Liczba '0'.
\end{itemize}
\item Działanie:
\begin{itemize}
\item Funkcja steruje działaniem modułu \verb+Simulation_test+. Wywołuje funkcję testową 
\verb+ should_show_generation_after_+\\\verb+simulation.+
\end{itemize}
\end{itemize}


\end{enumerate}
\subsection{Png test}
Jest modułem testującym działanie modułu \verb+Png_generator+. Zawiera następujące funkcje:
\begin{enumerate}

\item \verb+void should_show_confirmation_for_successful_png_generation+\\\verb+(char *filename, board_t b)+
\begin{itemize}
\item Argumenty:
\begin{itemize}
\item \verb+filename+ - nazwa pliku png,
\item \verb+b+ - dane w strukturze \verb+board_t+ zawierające planszę przeznaczoną do zapisania w png,
\end{itemize}
\item Wartość zwracana:
\begin{itemize}
\item Void.
\end{itemize}
\item Działanie:
\begin{itemize}
\item Funkcja próbuje wygenerować plik png o nazwie podanej jako argument na podstawie podanej planszy za pomocą funkcji \verb+process_file+ oraz \verb+generate+ z modułu \verb+Png_generator+.
\end{itemize}
\end{itemize}

\item \verb+void should_show_error_message_for_unsuccessful_png_generation+\\\verb+(char *filename, board_t b)+
\begin{itemize}
\item Argumenty:
\begin{itemize}
\item \verb+filename+ - nazwa pliku png,
\item \verb+b+ - dane w strukturze \verb+board_t+ zawierające planszę przeznaczoną do zapisania w png,
\end{itemize}
\item Wartość zwracana:
\begin{itemize}
\item Void.
\end{itemize}
\item Działanie:
\begin{itemize}
\item Funkcja próbuje wygenerować plik png o nazwie podanej jako argument na podstawie podanej planszy za pomocą funkcji \verb+process_file+ oraz \verb+generate+ z modułu \verb+Png_generator+. Próba ta powinna zakończyć się niepowodzeniem i wypisaniem komunikatu o błędzie.
\end{itemize}
\end{itemize}

\item \begin{verbatim} int main(int argc, char** argv) \end{verbatim}
\begin{itemize}
\item Argumenty:
\begin{itemize}
\item \verb+argc+ - ilość podanych argumentów w wierszu poleceń ,
\item \verb+argv+ - argumenty podane przez użytkownika w wierszu poleceń.
\end{itemize}
\item Wartość zwracana:
\begin{itemize}
\item Liczba '0'.
\end{itemize}
\item Działanie:
\begin{itemize}
\item Funkcja steruje działaniem modułu \verb+Png_test+. Wywołuje funkcję testową \verb+ should_show_confirmation_for_successful_+\\\verb+png_generation+ oraz \verb+void should_show_error_message_+\\\verb+for_unsuccessful_png_generation+.
\end{itemize}
\end{itemize}


\end{enumerate}

\section{Dane w naszym programie}
Przechowywanie danych, sposób ich przyjmowania i generowania przez nasz program został opisany w specyfikacji funkcjonalnej. Argumenty przyjmowane przez poszczególne funkcje oraz ich typy opisane są w sekcji \ref{Opis} Opis modułów i ich funkcji.

\section{Testy}
Opis funkcji testujących został opisany w poprzednim rozdziale podczas opisu modułów \verb+Board_test+,
\verb+Simulation_test+ oraz \verb+Png_test+.

Podczas testów będziemy analizować dane, które zostaną zwrócone przez funkcje testujące i na ich podstawie ocenimy działanie naszego programu.

Do sprawdzenia czy nasz program nie powoduje wycieków pamięci użyjemy programu \verb+valgrind+.

\section{Strona techniczna projektu}
\begin{description}
\item[Używana wersja C]\hfill \\
 W naszym projekcie zostanie użyty język C w wersji~89.
\item[Używany system operacyjny]\hfill \\
 Projekt będzie tworzony w systemie Linux.
\item[Użyte biblioteki]\hfill \\
 Nasz program będzie korzystał z biblioteki \verb+libpng+.
\item[Wersjonowanie]\hfill \\
 Podczas tworzenia projektu będziemy go wersjonować zaczynając od \verb+0.1+, a gotowa wersja naszego programu będzie miała numer \verb+1.0+. Podczas pracy z Gitem będziemy tworzyć nowy branch za każdym razem gdy będziemy chcieli wprowadzić nową funkcjonalność.



\end{description}


\end{document}